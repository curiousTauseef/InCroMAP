\documentclass{bioinfo}
\copyrightyear{2012}
\pubyear{2012}

\begin{document}
\firstpage{1}

\title[Pathway-based visualization of cross-platform microarray datasets]{Pathway-based visualization of cross-platform microarray datasets}
\author[Clemens Wrzodek \textit{et~al}]{Clemens Wrzodek\,$^{1\footnote{to whom correspondence should be addressed}}$, Johannes Eichner\,$^{1}$ and Andreas Zell\,$^1$}

\address{$^{1}$Center for Bioinformatics Tuebingen (ZBIT), \\University of Tuebingen, 72076 T\"ubingen, Germany}
\history{Received on XXXXX; revised on XXXXX; accepted on XXXXX}

\editor{Associate Editor: XXXXXXX}

\maketitle

\begin{abstract}

\section{Motivation:}
Text Text Text  Text Text Text Text Text Text Text Text
Text  Text Text Text Text Text Text Text Text Text  Text Text Text Text Text Text Text Text Text  Text Text Text Text Text Text Text Text Text  Text Text Text Text Text Text Text Text Text  Text Text Text Text Text Text Text Text Text  Text Text Text Text Text.

\section{Results:}
Text  Text Text Text Text Text Text Text Text Text  Text Text Text Text Text Text Text Text Text  Text Text Text Text Text Text Text Text Text  Text Text Text Text Text Text

%\section{Availability:}

\section{Contact:} \href{clemens.wrzodek@uni-tuebingen.de}{clemens.wrzodek@uni-tuebingen.de}
\end{abstract}

\section{Introduction}

During the first years after the invention of microarrays, researchers focused their experiments largely on messenger RNA (mRNA) expression data. Therefore, many analysis and visualization methods have been developed especially for mRNA datasets. Until today, multiple different microarray platforms have been developed. These different platforms include, for example, microarrays to measure the expression of micro RNAs (miRNAs), proteins, or even the amount of DNA methylation in defined genomic regions  [TODO - CITE REVIEW PAPER, SIEHE SEMINAR!]. For all of those platforms, preprocessing, analysis and visualization methods also became available. But, until today, methods that can integratively visualize multiple microarray datasets, coming from multiple different platforms are very rare.

We introduce a method, for integrated pathway-based visualization of multiple different microarray datasets. 

% Warum ist PW-basierte visualisierung, grade f�r mehrere Datentypen, besser als UCSC genome browser oder anderes.

Microarrays are probe-based and these probes are mostly not spread randomly across the genome. Probes are usually picked for interesting genes or regions, which includes genes, that are important for certain pathways. Region-based visualization methods, like the USCS Genome Browser [TODO CITE], may be suitable for visualization if researchers already know some regions of interest. But in general, they are not good visualization methods for probe-based datasets. 

To get a starting point for microarray analysis and because microarrays usually contain probes for most important pathways, the popular pathway enrichment analysis is very often performed on microarray datasets. And this is already the ending point of most pathway analysis methods. Some applications offer users to show the pathway enrichments as bar plots, but visualizing the pathways and plotting the microarray data directly in the pathway are rare features. For these reasons, we show how to integratively visualize data from multiple different microarray datasets in a pathway.

% TODO: Erw�hnen dass mRNA knoten einf�rbung ein "alter hut ist", aber das besodere bei den anderen 3 datentypen ist.


We present a novel method that includes visualizing pathways and changing the pathway to reflect expression data from mRNA, miRNA, DNA methylation and (phospho-) protein datasets.


% The amount of available expression datasets is growing rapidly from day to day. But not only the

There are other tools, specialized in pathway analysis (e.g., Ingenuity, ), or in pathway visualization (Cytoscape, KEGG Atlas). Some even offer visualizing data in a pathway (GenMAPP, KEGG Array, Symony programm). 

TODO: das h�chste der gef�hle sind methoden um 2 farben in 1 knoten, aber keiner kann DNA methylation oder sogar miRNA knoten rein.

TODO Johannes: Kannst du 1-2 S�tze jeweils zu Cytoscape, Ingenuity, eventuell noch Cytoscape + Plugins schreiben?


Related Work. Abgrenzung zu GenMAPP und Ingenuity, Cytoscape, KEGG Atlas, KEGG Array; Siehe auch Kohlbacher-Nils Paper, S. Symons paper (niselt).
Gibt es ueberhaupt ein Tool, welches alle 4 Datentypen (miRNA, etc) visualisieren kann?


%\begin{methods}
\section{Methods}
Einleitender, kurzer Abschnitt, welcher auf Figure mit Alle-daten-in-1-graph zeigt.

\subsection{Pathway visualization}
Auf KEGGtranslator und generell (auch mehrere gene in 1 knoten, etc) Pathway visualisierung eingehen.

\textbf{Clemens.}

\subsection{Visualization of messenger RNA expression data}
1 wert berechnen, Knotenfarbe

\subsection{Visualization of micro RNA expression data}
erst auf Targets eingehen.

Dann wie targets in graph gef�gt werden

Dann wie knoten eingef�rbt werden.

\subsection{Visualization of protein and protein modification expression data}

Erkl�ren, warum man modifikationen nicht mischen darf; Auf box-visualisierung eingehen; erw�hnen, dass auch normale protein daten visualisiert werden k�nnen.

\textbf{Johannes.}

\subsection{Visualization of DNA methylation data}

Ein Wert nur als Hinweis, hier geht etwas [click gibt details?]. fold-change wird zu box von -2 bis +2, p-value im grunde ein bar-blot von 1 bis 0.00005 oder so...

Einzelner Wert mit binning und $\frac{\sum\limits_{i=1}^n\log_2 x}{n}$, f�r fold-changes oder so peak detection m�glich und max. peak anzeigen.

\textbf{Johannes.}


\subsection{OFFENE FRAGEN}
Sollten wir hier einfaerbung nach enrichment p-values bzw. den "metabolic pathways"-pathway erwaehnen? Oder lieber fuer spaetere publikationen "aufspaaren"?

%\end{methods}



\section{Results and discussion}

Gesamtkonzept und Ergebnisse / Bilder vorstellen

hier erw�hnen, dass methoden in InCroMAP drin sind? oder lieber in conslutions?






\begin{figure}[!tpb]%figure1
%\centerline{\includegraphics{fig01.eps}}
\caption{Caption, caption.}\label{fig:01}
\end{figure}

\begin{figure}[!tpb]%figure2
%\centerline{\includegraphics{fig02.eps}}
\caption{Caption, caption.}\label{fig:02}
\end{figure}





%%%%%%%%%%%%%%%%%%%%%%%%%%%%%%%%%%%%%%%%%%%%%%%%%%%%%%%%%%%%%%%%%%%%%%%%%%%%%%%%%%%%%
%
%     please remove the " % " symbol from \centerline{\includegraphics{fig01.eps}}
%     as it may ignore the figures.
%
%%%%%%%%%%%%%%%%%%%%%%%%%%%%%%%%%%%%%%%%%%%%%%%%%%%%%%%%%%%%%%%%%%%%%%%%%%%%%%%%%%%%%%






\section{Conclusion}

TODO



\section*{Acknowledgement}
Text Text Text Text Text Text  Text Text.  \citealp{Boffelli03} might want to know about  text text text text

\paragraph{Funding\textcolon} The research leading to these results has received funding from the Innovative Medicine Initiative Joint Undertaking (IMI JU) under grant agreement nr. 115001 (MARCAR project).

%\bibliographystyle{natbib}
%\bibliographystyle{achemnat}
%\bibliographystyle{plainnat}
%\bibliographystyle{abbrv}
%\bibliographystyle{bioinformatics}
%
%\bibliographystyle{plain}
%
%\bibliography{Document}


\begin{thebibliography}{}
\bibitem[Bofelli {\it et~al}., 2000]{Boffelli03} Bofelli,F., Name2, Name3 (2003) Article title, {\it Journal Name}, {\bf 199}, 133-154.

\bibitem[Bag {\it et~al}., 2001]{Bag01} Bag,M., Name2, Name3 (2001) Article title, {\it Journal Name}, {\bf 99}, 33-54.

\bibitem[Yoo \textit{et~al}., 2003]{Yoo03}
Yoo,M.S. \textit{et~al}. (2003) Oxidative stress regulated genes
in nigral dopaminergic neurnol cell: correlation with the known
pathology in Parkinson's disease. \textit{Brain Res. Mol. Brain
Res.}, \textbf{110}(Suppl. 1), 76--84.

\bibitem[Lehmann, 1986]{Leh86}
Lehmann,E.L. (1986) Chapter title. \textit{Book Title}. Vol.~1, 2nd edn. Springer-Verlag, New York.

\bibitem[Crenshaw and Jones, 2003]{Cre03}
Crenshaw, B.,III, and Jones, W.B.,Jr (2003) The future of clinical
cancer management: one tumor, one chip. \textit{Bioinformatics},
doi:10.1093/bioinformatics/btn000.

\bibitem[Auhtor \textit{et~al}. (2000)]{Aut00}
Auhtor,A.B. \textit{et~al}. (2000) Chapter title. In Smith, A.C.
(ed.), \textit{Book Title}, 2nd edn. Publisher, Location, Vol. 1, pp.
???--???.

\bibitem[Bardet, 1920]{Bar20}
Bardet, G. (1920) Sur un syndrome d'obesite infantile avec
polydactylie et retinite pigmentaire (contribution a l'etude des
formes cliniques de l'obesite hypophysaire). PhD Thesis, name of
institution, Paris, France.

\end{thebibliography}
\end{document}
